% Version 1.0.1 DE
%
% Seitenbegrenzungen anzeigen
%\usepackage{showframe}

%%%%%%%%%%%%%%%%%%%%%%%%%%%%%%%%%%%%%%%%%%%%%%%%%%
%%% 				Informationen			  %%%%
%%%%%%%%%%%%%%%%%%%%%%%%%%%%%%%%%%%%%%%%%%%%%%%%%%

% Zeilenumbrüche manuell in \listoffigures und \listoftables setzen mit "\protect\\" (ohne Anführungszeichen) im optionalen Parameter des "\caption"-Befehls
% Beispiel: 
%\caption[Exterior view of the KIT library. Consetetur sadipscing elitr,\protect\\ sed diam nonumy eirmod tempor invidunt ut labore]{Exterior view of the KIT library. Consetetur sadipscing elitr, sed diam nonumy eirmod tempor invidunt ut labore}

% Geschütztes Leerzeichen: "~" (ohne Anführungszeichen)

%%%%%%%%%%%%%%%%%%%%%%%%%%%%%%%%%%%%%%%%%%%%%%%%%%
%%% 	Allgemeine Einstellungen			  %%%%
%%%%%%%%%%%%%%%%%%%%%%%%%%%%%%%%%%%%%%%%%%%%%%%%%%

% Seitenränder einstellen
\setlength{\topskip}{10.5pt} % Verhindern einer Fehlermeldung (Zusatz: siehe Kohm 2020: S. 36, Tabelle 2.1 "Satzspiegelmaße in Abhängigkeit von DIV bei A4 ohne Berücksichtigung von \topskip oder BCOR)

%%%%%%%%%%%%%%%%%%%%%%%%%%%%%%%%%%%%%%%%%%%%%%%%%%
%%%     	Eingabe, Ausgabe, Umlaute 		   %%%
%%%%%%%%%%%%%%%%%%%%%%%%%%%%%%%%%%%%%%%%%%%%%%%%%%

% Laden der Sprachpakete (die optionalen Parameter "ngerman" für die neue deutsche Rechtschreibung und "english" sind bereits in den optionalen Parametern der "\documentclass" gesetzt)
\usepackage{babel}

% Eingabe von Umlauten (ä, ö, ü sowie ß) durch den Parameter [utf8]
\usepackage[utf8]{inputenc}

% Ausgabe von Umlauten
\usepackage[T1]{fontenc}

%%%%%%%%%%%%%%%%%%%%%%%%%%%%%%%%%%%%%%%%%%%%%%%%%%
%%%        		Skalierung					  %%%%
%%%%%%%%%%%%%%%%%%%%%%%%%%%%%%%%%%%%%%%%%%%%%%%%%%

% Bitte skalieren Sie, entsprechend der Seitenanzahl ihres Dokuments, die Seitenränder wie folgt:

% - bis 199 Seiten (innen: 20mm, außen: 15-18mm) >>> textwidth=113mm
% - 200 bis 399 Seiten (innen: 23mm, außen: 15-18mm) >>> textwidth=110mm
% - ab 400 Seiten (innen: 25mm, außen: 15mm) >>> textwidth=108mm

%%%%%%%%%%%%%%%%%%%%%%%%%%%%%%%%%%%%%%%%%%%%%%%%%%
%%%        		Papierformat (A5)			   %%%
%%%%%%%%%%%%%%%%%%%%%%%%%%%%%%%%%%%%%%%%%%%%%%%%%%

% Seitengröße auf "a4paper" oder "a5paper" einstellbar (1. oder 2. "usepackage" wählen)
% Hinweis: Die Seitenränder für die Skalierung (s. o. "Skalierung") kann man mithilfe des Pakets "geometry" verwalten, siehe bitte hierzu die entsprechende Dokumentation unter: http://ftp.fau.de/ctan/macros/latex/contrib/geometry/geometry.pdf

% (1)
\usepackage[%
a5paper,%
headheight=1.5\baselineskip,%
top=25mm,%
inner=20mm,%
outer=15mm,%
%lines=38,%
%textwidth=113mm,%
footnotesep=7mm,%
heightrounded=true%
]{geometry}

% (2)
%\usepackage[
%a4paper,
%headheight=1.5\baselineskip,
%top=25mm,
%lines=46,
%textwidth=160mm,
%heightrounded=true,
%%bindingoffset=15mm
%]{geometry}

%\usepackage[a4paper,top=33mm,bottom=32mm,outer=25mm,inner=35mm]{geometry}

%%%%%%%%%%%%%%%%%%%%%%%%%%%%%%%%%%%%%%%%%%%%%%%%%%
%%%        			Fußzeile			   	   %%%
%%%%%%%%%%%%%%%%%%%%%%%%%%%%%%%%%%%%%%%%%%%%%%%%%%

% Abstand zwischen Textkörper und Unterkante Fußzeile (Seitenzahlen)
\setlength{\footskip}{10mm}

% Abstand zwischen Fließtext und Fußnotentrennlinie 
\setlength{\skip\footins}{20pt}

%%%%%%%%%%%%%%%%%%%%%%%%%%%%%%%%%%%%%%%%%%%%%%%%%%
%%%        		Fußnote					   	   %%%
%%%%%%%%%%%%%%%%%%%%%%%%%%%%%%%%%%%%%%%%%%%%%%%%%%

% (1) Position und Größe der Fußnoten bei 2-stelligen Fußnotennummern
\deffootnote[1.8em]{1.8em}{0em}{\makebox[1.7em][l]{\textsuperscript{\thefootnotemark\ }}}

% (2) Bitte wählen Sie für 3-stellige Fußnotennummern den folgenden Befehl und deaktivieren Sie (1), indem Sie den vorhergehenden Befehl "\deffootnote{1.7em}{0em}{\makebox[1.7em][l]{\thefootnotemark}}" auskommentieren
%\deffootnote{2.2em}{0em}{\makebox[2.2em][l]{\thefootnotemark}}

% Verhindert das Fortsetzen von Fussnoten auf der gegenüberliegenden Seite
\interfootnotelinepenalty=10000 

%%%%%%%%%%%%%%%%%%%%%%%%%%%%%%%%%%%%%%%%%%%%%%%%%%
%%%  				Absatz				  	   %%%
%%%%%%%%%%%%%%%%%%%%%%%%%%%%%%%%%%%%%%%%%%%%%%%%%%

% Zeilen auf der Seite verteilen (Es wird kein Ausgleich des unteren Seitenrandes durch Dehnung der Absatzabstände durchgeführt) 
\raggedbottom   

% Einzugtiefe (horizontaler Abstand) der ersten Zeile des Absatzes
\setlength{\parindent}{0pt}

% Vertikaler Abstand zwischen den Absätzen
\setlength{\parskip}{2.5mm}

% Schusterjungen (einzelne Zeile unten auf der Seite) unterdrücken
\clubpenalty = 10000 

% Hurenkinder (einzelne Zeile oben auf der Seite) unterdrücken
\widowpenalty = 10000
\displaywidowpenalty = 10000

% Silbentrennung am Seitenumbruch verhindern
\brokenpenalty = 10000

%%%%%%%%%%%%%%%%%%%%%%%%%%%%%%%%%%%%%%%%%%%%%%%%%%
%%%  			Gleitobjekte				   %%%
%%%			Abbildungen	Tabellen			   %%%
%%%			"{figure}", "{table}"		   	   %%%
%%%		Beschriftungen, Abstand, Caption  	   %%%
%%%%%%%%%%%%%%%%%%%%%%%%%%%%%%%%%%%%%%%%%%%%%%%%%%

\usepackage[
labelfont=bf, % Fette Beschriftungen
font=footnotesize, % Schriftgröße für Beschriftungen
]
{caption}

% Abbildungen
\captionsetup[figure]{position=below} 
\captionsetup[figure]{aboveskip=3mm} %  Abstand Bild-Bildunterschrift
\captionsetup[figure]{belowskip=-2mm} % Abstand Bildunterschrift-Fließtext

% Tabellen
% Bei der Nutzung von "\captionabove" werden die Werte "aboveskip" und "belowskip" vertauscht; bitte nutzen Sie daher den Befehl "\caption" für Tabellenüberschriften
\captionsetup[table]{position=top} 
\captionsetup[table]{aboveskip=1.5mm} % Abstand: Fließtext-Tabellenüberschrift (im PDF gemessen=~7mm) | Anmerkung: mit dem Befehl "\captionabove" ändert sich die Reihenfolge der Tabellenüberschrift  in: Tabellenüberschrift-Tabelle
\captionsetup[table]{belowskip=2.5mm} % Abstand: Tabellenüberschrift-Tabelle (im PDF gemessen=~3mm)| Anmerkung: mit dem Befehl "\captionabove" ändert sich die Reihenfolge der Tabellenüberschrift in: Fließtext-Tabellenüberschrift

% Abstand Bild: Fließtext-Bild; Bildunterschrift-Fließtext
% Abstand Tabelle: Fließtext-Tabellenüberschrift; Tabelle-Fließtext
\setlength{\intextsep}{5.5mm plus0mm minus0mm} % (im PDF gemessen für Tabellen=~7mm; für Abbildungen=~5mm)
\setlength{\textfloatsep}{7mm plus0mm minus0mm}

% Bildunterschrift Subfloats
\captionsetup[subfloat]{%
	labelformat = empty,%
	margin = 0pt,		% Einzug der Bildunterschrift von links
	%	skip = 0pt,		% Abstand zwischen Bild und Bildunterschrift
	aboveskip = 2mm,	% Abstand zwischen Bild und Bildunterschrift
	belowskip = 0mm,	% Abstand zwischen Bildunterschrift und der nächsten Bildreihe sowie zwischen Bildunterschrift und Unterschrift der Abbildung / da es hier zu einer Überschneidung der Befehle "\captionsetup[subfloat]{belowskip}" und "\captionskip[figure]{skip}" kommt, muss der vertikale Abstand zwischen den Bildreihen mit dem Befehl "\vspace{3mm}" gesetzt werden
	%	font = {footnotesize,rm},		%
	%	labelfont = {footnotesize,bf},	%
	format = hang, 		% Zweite und weitere Zeilen einrücken (an die erste ausrichten)
	indention = 0em,	% Einruecken der Beschriftung
	labelsep = space,	% "period, space, quad, newline"
	justification = RaggedRight,	% Flattersatz mit Silbentrennung
	%	justification = raggedright,	% Flattersatz ohne Silbentrennung
	%	justification = centering,	% Zentriert
	%	justification = justified,	% Blocksatz
	singlelinecheck = true, % "false" (true = bei einer Zeile immer zentrieren)
	position = auto,		% "top, bottom"
	labelformat = parens	% "simple, empty" = wie die Bezeichnung gesetzt wird
}

%%%%%%%%%%%%%%%%%%%%%%%%%%%%%%%%%%%%%%%%%%%%%%%%%%
%%%  			Gleitobjekte				   %%%
%%%			"{figure}", "{table}"		   	   %%%
%%%				Layout, Größe  	   			   %%%
%%%%%%%%%%%%%%%%%%%%%%%%%%%%%%%%%%%%%%%%%%%%%%%%%%

% Mindestfüllgrad einer Seite mit einem Gleitobjekt
\renewcommand{\floatpagefraction}{0.7}

% Maximale Größe eines Gleitobjekts am unteren Seitenrand
\renewcommand{\topfraction}{0.8}

% Maximale Größe eines Gleitobjekts am oberen Seitenrand
\renewcommand{\bottomfraction}{0.8}

% Mögliche Abstandsvergrößerung innerhalb einer Zeile bei unschönem Zeilenumbruch
\setlength{\emergencystretch}{4pt}

% Mindestanteil an Text auf einer Seite mit Gleitobjekt
\renewcommand{\textfraction}{0.1}

%%%%%%%%%%%%%%%%%%%%%%%%%%%%%%%%%%%%%%%%%%%%%%%%%%
%%%  				Zeilenabstand			   %%%
%%%%%%%%%%%%%%%%%%%%%%%%%%%%%%%%%%%%%%%%%%%%%%%%%%

% Zeilenabstände auf 1-fach festlegen
%\usepackage[singlespacing]{setspace}

% Zeilenabstände auf 1,15-fach festlegen
\usepackage{setspace}
\setstretch{1.15}

% Zeilenabstände auf 1,2-fach festlegen
%\usepackage{setspace}
%\setstretch{1.2}

% Zeilenabstände auf 1,5-fach festlegen
%\usepackage[onehalfspacing]{setspace}

%%%%%%%%%%%%%%%%%%%%%%%%%%%%%%%%%%%%%%%%%%%%%%%%%%
%%% 		Inhaltsverzeichnis		   		   %%%
%%%%%%%%%%%%%%%%%%%%%%%%%%%%%%%%%%%%%%%%%%%%%%%%%%

% Inhaltsverzeichnis richtig darstellen
\usepackage[titles]{tocloft}

% Auch bei den Kapiteln Punkte darstellen
\renewcommand{\cftchapdotsep}{\cftdotsep}
\renewcommand{\cftchapleader}{\cftdotfill{\cftchapdotsep}}

% Seitenzahlen bei Kapitel in serifenloser Schriftart darstellen
\renewcommand{\cftchappagefont}{\fontfamily{phv}\normalsize\bfseries}

% Verzeichnisse aktualisieren
%Fonts im Inhaltsverzeichnis
\renewcommand\cftchapfont{\fontfamily{phv}\normalsize\bfseries}
\renewcommand\cftsecfont{\fontfamily{phv}\fontsize{11}{11}}

%Fonts in Kapiteln und sections...
%\renewcommand\cftchappagefont{\fontfamily{phv}\normalsize\bfseries}
\renewcommand\cftsecpagefont{\fontfamily{phv}\fontsize{11}{11}}

\usepackage{makeidx}

%%%%%%%%%%%%%%%%%%%%%%%%%%%%%%%%%%%%%%%%%%%%%%%%%%
%%% 			Überschriften			  	   %%%
%%%%%%%%%%%%%%%%%%%%%%%%%%%%%%%%%%%%%%%%%%%%%%%%%%

% Auch die 4. Ebene nummerieren (subsubsection)
\setcounter{secnumdepth}{4} 

% Bis Ebene 3 (subsection) im Inhaltsverzeichnis anzeigen
\setcounter{tocdepth}{2}

% Schriftarten und -größen für die Überschriften vorgeben
\addtokomafont{chapter}{\fontfamily{phv}\fontsize{18}{20}\bfseries} 		% z. B. "2 Stand der Technik"
\addtokomafont{section}{\fontfamily{phv}\fontsize{14}{16}\bfseries}			% z. B. "2.1 Literatur und Forschung"
\addtokomafont{subsection}{\fontfamily{phv}\fontsize{12}{14}\bfseries}		% z. B. "2.1.1 Disziplinäre Entwicklung"
\addtokomafont{subsubsection}{\fontfamily{phv}\fontsize{10}{12}\bfseries}	% z. B. "2.1.1.1 Genese wissenschaftlicher" Konzepte"

%%%%%%%%%%%%%%%%%%%%%%%%%%%%%%%%%%%%%%%%%%%%%%%%%%
%%% 		Überschriften auslinieren 	   	   %%%
%%%%%%%%%%%%%%%%%%%%%%%%%%%%%%%%%%%%%%%%%%%%%%%%%%

% Horizonaler Abstand zwischen Nummerierung und Überschrift
\renewcommand*{\chapterformat}{\makebox[1.3cm][l]{\thechapter\autodot}}
\renewcommand*{\sectionformat}{\makebox[1.3cm][l]{\thesection\autodot}}
\renewcommand*{\subsectionformat}{\makebox[1.3cm][l]{\thesubsection\autodot}}
\renewcommand*{\subsubsectionformat}{\makebox[1.3cm][l]{\thesubsubsection\autodot}}

%%%%%%%%%%%%%%%%%%%%%%%%%%%%%%%%%%%%%%%%%%%%%%%%%%
%%% 	Bezeichnungen: Abbildung / Tabelle 	   %%%
%%%%%%%%%%%%%%%%%%%%%%%%%%%%%%%%%%%%%%%%%%%%%%%%%%

% Bezeichnungsnamen angeben
\addto\captionsngerman{\renewcommand{\figurename}{Abbildung}}
\addto\captionsngerman{\renewcommand{\tablename}{Tabelle}}

%%%%%%%%%%%%%%%%%%%%%%%%%%%%%%%%%%%%%%%%%%%%%%%%%%
%%% 				Schriftgröße			   %%%
%%%			Abbildungen, Kopf- Fußzeilen, 	   %%%
%%%		 	Seitenzahl,Textfarbe   	   		   %%%
%%%%%%%%%%%%%%%%%%%%%%%%%%%%%%%%%%%%%%%%%%%%%%%%%%

% Größen der Beschriftungen vorgeben
%\addtokomafont{caption}{\footnotesize}
%\setkomafont{captionlabel}{\footnotesize}

% Größe der Kopf- und Fußzeile vorgeben
\setkomafont{pageheadfoot}{\footnotesize} 

% Größe der Seitenzahl
\setkomafont{pagenumber}{\normalsize}

% Farben im Dokument zulassen
\usepackage{color}

% Textfarbe schwarz definieren
\color[cmyk]{0,0,0,1}

%%%%%%%%%%%%%%%%%%%%%%%%%%%%%%%%%%%%%%%%%%%%%%%%%%
%%% 			Schriftarten			  	   %%%
%%%%%%%%%%%%%%%%%%%%%%%%%%%%%%%%%%%%%%%%%%%%%%%%%%

%%%%%%%%%%%%%%%%%%%%%%%%%%%%%%%%%%%%%%%%%%%%%%%%%%
%%% 			Mit Serifen			  	   	   %%%
%%%%%%%%%%%%%%%%%%%%%%%%%%%%%%%%%%%%%%%%%%%%%%%%%%

% Nimbus 15 Serif
% Beispiele zum Schriftbild (Typografie)siehe: https://tug.org/FontCatalogue/nimbus15serif/
\usepackage{nimbusserif}

% URW Nimbus Roman (ähnlich Times New Roman)
% Beispiele zum Schriftbild (Typografie)siehe: https://tug.org/FontCatalogue/urwnimbusroman/ 
%\usepackage{mathptmx}

% Utopia Regular with Fourier
% Beispiele zum Schriftbild (Typografie)siehe: https://tug.org/FontCatalogue/utopia-fouriermath/ 
%\usepackage{fourier}

% Utopia Regular with Math Design
% Beispiele zum Schriftbild (Typografie)siehe: https://tug.org/FontCatalogue/utopia-mathdesign/ \usepackage[adobe-utopia]{mathdesign}

%%%%%%%%%%%%%%%%%%%%%%%%%%%%%%%%%%%%%%%%%%%%%%%%%%
%%% 			Ohne Serifen		  	   	   %%%
%%%%%%%%%%%%%%%%%%%%%%%%%%%%%%%%%%%%%%%%%%%%%%%%%%

% URW Nimbus Sans
% Beispiele zum Schriftbild (Typografie)siehe: https://tug.org/FontCatalogue/urwnimbussans/ 
%\usepackage[scaled]{helvet}
%\renewcommand*\familydefault{\sfdefault}

% Nimbus 15 Sans
% Beispiele zum Schriftbild (Typografie)siehe: https://tug.org/FontCatalogue/nimbus15sans/ 
%\usepackage{nimbussans}
%\renewcommand*\familydefault{\sfdefault}

%%%%%%%%%%%%%%%%%%%%%%%%%%%%%%%%%%%%%%%%%%%%%%%%%%
%%%        			Microtype			   	   %%%
%%%%%%%%%%%%%%%%%%%%%%%%%%%%%%%%%%%%%%%%%%%%%%%%%%
% \usepackage[stretch=10,shrink=10]{microtype} % Verhindert Unschärfe und Verschwimmen der Schrift und reduziert die Anzahl der Badboxes (underfull/overfull); muss nach der Schriftart eingebunden werden

%%%%%%%%%%%%%%%%%%%%%%%%%%%%%%%%%%%%%%%%%%%%%%%%%%
%%%        			Kopfzeile			   	   %%%
%%%%%%%%%%%%%%%%%%%%%%%%%%%%%%%%%%%%%%%%%%%%%%%%%%

% Paket für Kopf- und Fußzeilen
\usepackage[
markcase=ignoreuppercase,
automark,
autooneside=false
]{scrlayer-scrpage}
%\usepackage[nouppercase]{scrpage2} % Obsoletes Paket, das durch das Paket "\usepackage{scrlayer-scrpage}" ersetzt wurde

% Linie in der Kopfzeile definieren 
% "Das Element [headsepline] wird nach \normalfont und nach den Elementen pageheadfoot und pagehead angewandt." (Kohm 2020: 268)
\KOMAoptions{headsepline=0.5pt}
%\setheadsepline{0.5pt} % Befehlszeile für das obsolete Paket "\usepackage{scrpage2}"

% Keine Kopf/Fußzeile auf leeren Seiten
%\usepackage{emptypage} % Usage of package `emptypage' together(scrbook) with a KOMA-Script class is not recommended

% Abstand zwischen Textkörper und Linie in der Kopfzeile
\setlength{\headsep}{8mm}

%%%%%%%%%%%%%%%%%%%%%%%%%%%%%%%%%%%%%%%%%%%%%%%%%%
%%% 				Notizen		  	   	   	   %%%
%%%%%%%%%%%%%%%%%%%%%%%%%%%%%%%%%%%%%%%%%%%%%%%%%%

% Erlaubt das Einfügen von Notizen
%\usepackage{todonotes}

% Deaktiviert alle eingefügten Notizen
%\usepackage[disable]{todonotes}

% Einfügen von Notizen im Fließtext
%\todo[inline]{Das ist eine Notiz}

%%%%%%%%%%%%%%%%%%%%%%%%%%%%%%%%%%%%%%%%%%%%%%%%%%
%%% 			URLs/links					   %%%
%%%%%%%%%%%%%%%%%%%%%%%%%%%%%%%%%%%%%%%%%%%%%%%%%%

% Darstellung URLs mit Zeilenumbruch
\usepackage[hyphens]{url}

% Zeilenumbrüche in URLs nach folgenden Zeichen
\appto\UrlBreaks{\do\a\do\b\do\c\do\d\do\e\do\f\do\g\do\h\do\i\do\j\do\k\do\l\do\m\do\n\do\o\do\p\do\q\do\r\do\s\do\t\do\u\do\v\do\w\do\x\do\y\do\z\do\/\do\.}

% Darstellung und Verlinkungen im pdf-Dokument einstellen
\usepackage[hidelinks,% Links als normaler Text darstellen
	pdfpagemode = UseNone,% Lesezeichen im PDF-Reader nicht anzeigen
	pdfpagelayout = TwoColumnRight,% Seitenanzeige des PDF-Dokuments angeben
	pdfauthor = {\autor},% Autor des PDF-Dokuments
	pdftitle = {\pdftitle}]% Titel des PDF-Dokuments
	{hyperref}
	
%%%%%%%%%%%%%%%%%%%%%%%%%%%%%%%%%%%%%%%%%%%%%%%%%%
%%% 	Einstellungen weiterer Pakete		   %%%
%%%%%%%%%%%%%%%%%%%%%%%%%%%%%%%%%%%%%%%%%%%%%%%%%%

% Ansicht Literaturverzeichnis
%\bibliographystyle{plaindin}

% Einbinden von Bildern ermöglichen
\usepackage{graphicx}	

% Gedrehte Objekte ermöglichen
\usepackage{rotating}

% Erweiterte Tabellenumgebung
\usepackage{tabularx}

% Erweiterte Flattersatz-Kommandos
\usepackage{ragged2e}

% Linksbündiger Flattersatz in den Bezeichnungen
%\usepackage[justification=RaggedRight]{caption}
%\usepackage[justification=justified]{caption}
%\captionsetup[subfigure]{justification=RaggedRight}

% Neuer Spaltentyp "L" mit Breitenangabe für linksbündigen Flattersatz
\newcolumntype{L}[1]{>{\RaggedRight\arraybackslash}p{#1}}

% Mathematische Symbole
\usepackage{amsmath,amssymb}

% Zeilen in Tabellen können verbunden werden
\usepackage{multirow}

% Zusätzliche Textsymbole zur Verfügung stellen
\usepackage{textcomp}

% Operatorensymbole definieren
\newcommand{\real}{\operatorname{Re}}				% Realteil
\newcommand{\opdiv}{\operatorname{div}}				% Divergenzoperator
\newcommand{\rot}{\operatorname{rot}}				% Rotationsoperator
\newcommand{\grad}{\operatorname{grad}}				% Gradientenoperator
\newcommand{\imag}{\operatorname{Im}}				% Imaginärteil
\newcommand{\imein}{\operatorname{j}}				% Imaginäre Einheit "j"

% Erweiterte Listenanweisungen
\usepackage{etoolbox}

% Einzug im Abbildungsverzeichnis zu null setzen
\renewcommand{\cftfigindent}{0cm}

% Einzug im Tabellenverzeichnis zu null setzen
\renewcommand{\cfttabindent}{0cm}
	
% Deutsches Abkürzungsverzeichnis erstellen
%\usepackage[english]{nomencl}
\usepackage[german]{nomencl}

% Befehl für einen Eintrag im Abkürzungsverzeichnis in "\sym" umbennen
\let\sym\nomenclature

% Name des Abkürzungsverzeichnis ändern
\renewcommand{\nomname}{Abkürzungs- und Symbolverzeichnis}

% Spaltenbreite der Formelzeichen auf "20 %" der Textbreite setzen
\setlength{\nomlabelwidth}{.2\textwidth}

% Einheiten in die Bezeichnung mit aufnehmen und rechtsbündig setzen
\newcommand{\nomunit}[1]{\renewcommand{\nomentryend}{\hspace*{\fill}#1}}

% Zeilenabstände verkleineren auf normalen Textabstand
\setlength\nomitemsep{-\parsep}

% Abkürzungsverzeichnis erzeugen
\makenomenclature

% Weitere Verzeichnisse erzeugen
\makeindex

\AtBeginDocument{ % 
  \newcaptionname{ngerman}\equationname{Formel} % 
  \newcaptionname{ngerman}\listequationname{Formelverzeichnis} % 
  \addtocontents{toc}{\protect\activateonlyattoc} % Z. B. für Umbrüche von langen Überschriften im Inhaltsverzeichnis mit dem Befehl \onlyattoc{\protect\\} (Beispiel: \chapter{Lange Kapitelüberschriften und der manuelle Zeilenumbruch \onlyattoc{\protect\\} für eine saubere Darstellung im Text})
}

\DeclareRobustCommand*{\onlyattoc}[1]{} % Z. B. für Umbrüche von langen Überschriften im Inhaltsverzeichnis mit dem Befehl \onlyattoc{\protect\\} (Beispiel: \chapter{Lange Kapitelüberschriften und der manuelle Zeilenumbruch \onlyattoc{\protect\\} für eine saubere Darstellung im Text})
\newcommand*{\activateonlyattoc}{\DeclareRobustCommand*{\onlyattoc}[1]{##1}}% Z. B. für Umbrüche von langen Überschriften im Inhaltsverzeichnis mit dem Befehl \onlyattoc{\protect\\} (Beispiel: \chapter{Lange Kapitelüberschriften und der manuelle Zeilenumbruch \onlyattoc{\protect\\} für eine saubere Darstellung im Text})

\DeclareNewTOC[ 
  tocentryindent=0pt,
  tocentrynumwidth=2em,
  type=equation,
  name={Gl.}, 
  types=equations, 
  listname={Formelverzeichnis}, 
]{equ} 
\newcommand{\equationentry}[2][\theequation]{ % 
  \addxcontentsline{equ}{equation}[{#1}]{\kern 1em #2} % 
} 
\BeforeStartingTOC[equ]{\def\autodot{:}}



% Literaturverzeichnis
%-----------------------
%
% Kohm 2020: Kohm, Markus; Neukamp, Frank; Kielhorn, Axel (2020): Die Anleitung. KOMA-Script. Markus Kohm. 2020-03.12. (Online)